\documentclass[11pt, a4paper, oneside]{article}

% URLs and hyperlinks ---------------------------------------
\usepackage{hyperref}
\hypersetup{
	colorlinks=true,
	linkcolor=blue,
	filecolor=magenta,      
	urlcolor=blue,
}
\usepackage[inline]{enumitem}
\usepackage{xurl}

\usepackage{caption}
%---------------------------------------------------

% page headers -------------------------------------------------
\usepackage{fancyhdr}
\fancypagestyle{plain}{\fancyhf{}\renewcommand{\headrulewidth}{0pt}}
\pagestyle{fancy}
\fancyhf{}% Clear header/footer
\fancyhead[L]{\nouppercase\leftmark}
\fancyhead[R]{\thepage}
%---------------------------------------------------------------

% adjust a verrrrry big table -------------------------------
\usepackage{adjustbox}
% -----------------------------------------------------------

% titlepage -------------------------------------------------
\usepackage{pdfpages}
%------------------------------------------------------------


% Rn custom item in enumeration -----------------------------
\newcounter{itemadded}
\setcounter{itemadded}{0}


\newcommand{\addeditem}{%
	\addtocounter{enumi}{-1}%
	\stepcounter{itemadded}
	\let\LaTeXStandardLabelEnumi\labelenumi%
	\addtocounter{enumi}{1}
	\renewcommand{\labelenumi}{\arabic{enumi}\lr{R}.}%
	\item 
	% Switch back to old labelling 
	\let\labelenumi\LaTeXStandardLabelEnumi%
}%


\let\LaTeXStandardEnumerateBegin\enumerate
\let\LaTeXStandardEnumerateEnd\endenumerate

\renewenvironment{enumerate}{%
	\LaTeXStandardEnumerateBegin%
	\setcounter{itemadded}{0}
}{%
	\LaTeXStandardEnumerateEnd%
}%
% -----------------------------------------------------------

% tables -------------------------------------------------------
\usepackage{float}
\usepackage{multirow}
\renewcommand{\arraystretch}{1.23}
% ---------------------------------------------------------------------

\usepackage{listings}
\usepackage{forest}
\usepackage{color}

\definecolor{dkgreen}{rgb}{0,0.6,0}
\definecolor{gray}{rgb}{0.5,0.5,0.5}
\definecolor{mauve}{rgb}{0.58,0,0.82}

\lstset{frame=tb,
	language=Lisp,
	aboveskip=3mm,
	belowskip=3mm,
	showstringspaces=false,
	columns=flexible,
	basicstyle={\small\ttfamily},
	numbers=none,
	numberstyle=\tiny\color{gray},
	keywordstyle=\color{blue},
	commentstyle=\color{dkgreen},
	stringstyle=\color{mauve},
	breaklines=true,
	breakatwhitespace=true,
	tabsize=3
}

\usepackage{xepersian}
\settextfont{Yas}
\setdigitfont{Yas}
\setlatintextfont{Yas}

\renewcommand{\lstlistlistingname}{فهرست کد‌ها}
\renewcommand{\bibname}{منابع}


\begin{document}
	\begin{titlepage}
		\centering
		\includegraphics[width=3.2cm, height=3.2cm]{images/logo}\par
		\vspace{5mm}
		{\LARGE دانشگاه اصفهان}\par
		\vspace{5mm}
		{\Large دانشکده مهندسی کامپیوتر}\par
		
		\vspace{2cm}
		
		{\Large تکلیف سری اول}\par
		
		\vspace{1cm}
		
		{\Large درس زبان‌های برنامه نویسی}\par
		
		\vspace{1cm}
		{\Huge آشنایی با زبان \lr{RUST}}\par
		
		
		\vspace{2cm}
		{\Large سیدمحمدحسین هاشمی}\par
		\vspace{1cm}
		{\large استاد : دکتر آرش شفیعی}\par

		\vspace{2cm}
		
		% Bottom of the page
		{\large پاییز 1403\par}
	\end{titlepage}
	
	\clearpage
	\begin{center}
		\includegraphics[width=10cm]{images/image002}
	\end{center}  
	\thispagestyle{plain}\mbox{} 
	\clearpage
	
	\tableofcontents
	%\listoftables 
	%\listoffigures
	%\lstlistoflistings
	\newpage
	
	\section{اسکیم در مقابل لیسپ}
	
		\begin{itemize}
			
			\item[الف.] چرا مجموعه کلمات کلیدی کوچک‌تر و مشخصات مختصر اسکیم می‌تواند مفید باشد؟
			
			\begin{itemize}
				
				\item {\large سادگی و یادگیری آسان‌تر}:
				{\normalsize تعداد کم کلمات کلیدی یادگیری زبان را برای مبتدیان ساده‌تر می‌کند.}
				
				\item {\large انعطاف‌پذیری و قابلیت بیان بالا}:
				{\normalsize توسعه‌دهندگان آزادی بیشتری برای ایجاد انتزاعات دارند بدون اینکه با تعداد زیادی از ساختارهای از پیش تعریف‌شده محدود شوند.}
				
				\item {\large تمرکز بر مفاهیم اصلی}:
				{\normalsize اسکیم بر سادگی تأکید دارد و برنامه‌نویسان را تشویق می‌کند تا مفاهیم اصلی مانند بازگشت (\lr{recursion}) و برنامه‌نویسی تابعی را بهتر درک کنند.}
				
			\end{itemize}
			
			\item[ب.] مثالی از تفاوت در رفتار با توجه به نوع محدوده (ایستا یا پویا):
			
			فرض کنید کد زیر نوشته شده است:
				\begin{latin}
					\begin{lstlisting}[caption={\lr{Q1 B}}]
(define x 10)  ;; global variable x

(define (foo) x)  ;; The function returns the value of x

(define (bar)
(let ((x 20))  ;; Local variable x
(foo)))      ;; calling foo

(bar)

					\end{lstlisting}
				\end{latin}
				
				\begin{itemize}
					
					\item {\large محدوده ایستا (\lr{Static Scoping})}:
					{\normalsize مقدار x در تابع \lr{foo} بر اساس جایی که \lr{foo} تعریف شده تعیین می‌شود، نه جایی که فراخوانی شده است. خروجی: 10 (مقدار سراسری \lr{x}).}
					
					\item {\large محدوده پویا (\lr{Dynamic Scoping})}:
					{\normalsize مقدار \lr{x} در تابع \lr{foo} بر اساس محیط فراخوانی \lr{bar} تعیین می‌شود. خروجی: 20 (مقدار محلی \lr{x} در \lr{bar}).}
					
				\end{itemize}
				
				محدوده پویا می‌تواند رفتار غیرقابل پیش‌بینی ایجاد کند، در حالی که محدوده ایستا قابل فهم‌تر و ساده‌تر برای اشکال‌زدایی است.
				
			\section{گام‌به‌گام ارزیابی عبارات در \lr{Lisp}}
			
				\begin{itemize}
					
					\item[الف.] گام‌به‌گام ارزیابی عبارات در \lr{Lisp}
					
						\begin{enumerate}
							
							\item عبارت اول:
							\begin{latin}
								\begin{lstlisting}[caption={\lr{Q2 A 1}}]
(eval '(+ 1 2 (eval '(+ 3 4))))

								\end{lstlisting}
							\end{latin}
							
							\begin{itemize}
								
								\item {\large گام اول} \par
								{\normalsize ارزیابی داخلی‌ترین عبارت \lr{(eval '(+ 3 4))}} \par
								{\normalsize عبارت \lr{(+ 3 4)} به دلیل نقل قول (\lr{quote}) به عنوان یک داده لیست به \lr{eval} داده می‌شود.} \par
								{\normalsize عبارت \lr{(+ 3 4)} به دلیل نقل قول (\lr{quote}) به عنوان یک داده لیست به \lr{eval} داده می‌شود.} \par
								
								\item {\large گام دوم} \par
								{\normalsize ارزیابی عبارت \lr{(+ 1 2 7)}} \par
								{\normalsize این عبارت یک جمع ساده است که نتیجه‌ی آن 10 خواهد بود.}
								
							\end{itemize}
							خروجی نهایی: 10.
							
							
							\item عبارت دوم:
							\begin{latin}
								\begin{lstlisting}[caption={\lr{Q2 A 2}}]
(cons 1 (list 2 3 (eval (cdr (cons 4 '(cdr '(5 6)))))))

								\end{lstlisting}
							\end{latin}
							
							\begin{itemize}
							
							\item {\large گام اول} \par
							{\normalsize ارزیابی \lr{(cons 4 '(cdr '(5 6)))}} \par
							{\normalsize \lr{cons} یک لیست می‌سازد که عنصر اول 4 و عنصر دوم \lr{'(cdr '(5 6))} به عنوان داده باقی می‌ماند.} \par
							{\normalsize نتیجه: \lr{(4 cdr '(5 6))}}
							
							\item {\large گام دوم} \par
							{\normalsize ارزیابی \lr{(cdr (cons 4 '(cdr '(5 6))))}} \par
							{\normalsize \lr{cdr} عنصر دوم لیست ساخته‌شده در مرحله قبل را برمی‌گرداند.} \par
							{\normalsize بنابراین خروجی \lr{cdr} برابر \lr{'cdr '(5 6)} می‌شود.} \par
							
							\item {\large گام سوم} \par
							{\normalsize \lr{eval} ابتدا \lr{cdr} را اعمال می‌کند بر روی لیست \lr{'(5 6)}.} \par
							{\normalsize \lr{cdr} عنصر دوم لیست را برمی‌گرداند که برابر با \lr{'(6)} است.} \par
							{\normalsize نتیجه: \lr{(eval (cdr '(5 6))) → '(6)}} \par
							
							\item {\large گام چهارم} \par
							{\normalsize \lr{list} عناصر \lr{2}، \lr{3} و \lr{'(6)} را در یک لیست قرار می‌دهد.} \par
							{\normalsize نتیجه: \lr{(2 3 (6))}} \par
							
							\item {\large گام پنجم} \par
							{\normalsize \lr{cons} عنصر \lr{1} را به ابتدای لیست \lr{(2 3 (6))} اضافه می‌کند.} \par
							
							\end{itemize}
							خروجی نهایی: \lr{(1 2 3 (6))}
							
							
						\end{enumerate}
					
					
					\item[ب.] پیاده‌سازی \lr{eval} در \lr{Lisp} با استفاده از نقل قول و پردازش لیست
					
					\lr{eval} را می‌توان در \lr{Lisp} با ترکیب قابلیت‌های زیر پیاده‌سازی کرد:
					
						\begin{enumerate}
							
							\item {\large نقل قول (\lr{Quoting})}:
							{\normalsize نقل قول (') به ما اجازه می‌دهد کد را به عنوان داده ذخیره کنیم و آن را به دلخواه پردازش کنیم.}
							
							\item {\large پردازش لیست‌ها (\lr{List Processing})}:
							{\normalsize از توابعی مانند \lr{car} (برای دسترسی به عنصر اول لیست)، cdr (برای دسترسی به بقیه عناصر لیست) و \lr{cons} (برای ساخت لیست) استفاده می‌کنیم.}
							
							\item {\large تطبیق الگو (\lr{Pattern Matching})}:
							{\normalsize برای شناسایی ساختار عبارات (مثلاً توابع و عملگرها) از تطبیق الگو بهره می‌گیریم.}
							
						\end{enumerate}
						
						\begin{latin}
							\begin{lstlisting}[caption={\lr{Q2 B - Pseudo code for eval}}]
(defun my-eval (expr)
(cond
((atom expr) expr)  ;; If expr is an atom (number, symbol), return itself
((equal (car expr) 'quote) (cadr expr))  ;; Quote mode
((equal (car expr) 'if)  ;; simple condition
(if (my-eval (cadr expr))
(my-eval (caddr expr))
(my-eval (cadddr expr))))
(t  ;; Mode of calling functions
(apply (my-eval (car expr)) (mapcar #'my-eval (cdr expr))))))

							\end{lstlisting}
						\end{latin}
						
						توضیح کد:
						
						\begin{enumerate}
							
							\item اگر \lr{expr} یک اتم باشد (عدد یا نماد)، خودش برگردانده می‌شود.
							
							\item اگر \lr{expr} یک نقل قول باشد، مقدار دوم لیست (داده‌ی نقل‌قول‌شده) برگردانده می‌شود.
							
							\item شرطی (\lr{if}): شرط بررسی و یکی از شاخه‌ها ارزیابی می‌شود.
							
							\item فراخوانی توابع:
							
							\begin{itemize}
								
								\item ابتدا تابع (عنصر اول لیست) ارزیابی می‌شود.
								
								\item سپس آرگومان‌های تابع با \lr{mapcar} ارزیابی می‌شوند.
								
								\item در نهایت، تابع با آرگومان‌ها اجرا می‌شود (\lr{apply}).
								
							\end{itemize}
							با این منطق، می‌توان \lr{eval} را به‌طور بازگشتی برای هر عبارت \lr{Lisp} پیاده‌سازی کرد.
							
							
						\end{enumerate}
						
				\end{itemize}
			
		\end{itemize}
		
		\section{کلمات کلیدی \lr{funcall} در لیسپ}
			
			\begin{itemize}
				
				\item[الف.] تابع \lr{mystery} شبیه به کدام تابع در \lr{ML} است؟
				
					تابع \lr{text} شبیه به تابع \lr{map} در زبان \lr{ML} است.
					
					\begin{itemize}
						
						\item در \lr{ML}، \lr{map} یک تابع مرتبه بالاتر (\lr{higher-order function}) است که یک تابع را به تمام عناصر لیست اعمال کرده و یک لیست جدید از نتایج ایجاد می‌کند.
						
						\item در \lr{mystery} نیز همین کار انجام می‌شود:
						
						\begin{itemize}
							
							\item هر عنصر از لیست (با استفاده از \lr{car} و \lr{cdr}) پردازش می‌شود.
							
							\item نتیجه تابع اعمال‌شده به عنصر، در یک لیست جدید جمع‌آوری می‌شود (با استفاده از \lr{cons}).
							
						\end{itemize}
						
					\end{itemize}
				
				\item[ب.] بازنویسی \lr{mystery} با استفاده از \lr{apply} به‌جای \lr{funcall}
				
				برای بازنویسی تابع \lr{mystery} با استفاده از \lr{apply} به‌جای \lr{funcall}، می‌توان به‌صورت زیر عمل کرد:
				
					\begin{latin}
						\begin{lstlisting}[caption={\lr{Q3 B}}]
(defun mystery (f x)
(cond
((consp x) (cons (apply f (list (car x)))  ;; به جای funcall از apply استفاده می‌شود
(mystery f (cdr x))))
(T nil)))

						\end{lstlisting}
					\end{latin}
					
					توضیح تغییرات:
					\begin{enumerate}
						
						\item در \lr{funcall f (car x)}، تابع \lr{f} مستقیماً با آرگومان (\lr{car x}) فراخوانی می‌شد.
						
						\item برای استفاده از \lr{apply}، نیاز است که آرگومان‌ها به‌صورت لیست ارائه شوند. بنابراین \lr{(list (car x))} ساخته می‌شود و به \lr{apply} داده می‌شود.
						
						\item \lr{apply} تابع \lr{f} را به همراه لیست آرگومان‌ها اجرا می‌کند.
						
					\end{enumerate}
					
					نکته: رفتار کد همچنان مشابه کد اصلی خواهد بود، با این تفاوت که از \lr{apply} به‌جای \lr{funcall} استفاده شده است.
		
			\end{itemize}
			
	\section{صریح‌سازی پرانتزها در عبارات $\lambda$}
	
		در حساب $\lambda$، پرانتزها برای تعیین ترتیب اعمال بسیار مهم هستند. ترتیب کلی به این شکل است:
		
		\begin{enumerate}
			
			\item چپ‌گرایی در کاربرد توابع
			
			\item اولویت بالاتر برای تعریف $\lambda$ در مقایسه با کاربرد تابع
			
		\end{enumerate}
		
		با این اصول، عبارات داده‌شده به‌طور صریح بازنویسی می‌شوند.
		
		\begin{enumerate}
			
			\item \lr{$\lambda$x.xz $\lambda$y.xy} \par
			
			گام به گام:
			
			\begin{itemize}
				
				\item ابتدا تعریف \lr{$\lambda$x} و کاربرد آن بر \lr{$\lambda$y.xy}
				
				\item در اینجا \lr{$\lambda$x.xz} یک تابع است که بر \lr{($\lambda$y.xy)} اعمال می‌شود.
				
			\end{itemize}
			
			بازنویسی کامل با پرانتزهای صریح:
			\[
			\left( \left( \lambda x . (x \, z) \right) \, \left( \lambda y . (x \, y) \right) \right)
			\]
			
			\item \lr{($\lambda$x.xz) $\lambda$y.w $\lambda$w.wyzx} \par
			
			گام به گام:
			
			\begin{itemize}
				
				\item ابتدا \lr{$\lambda$x.xz} بر \lr{($\lambda$y.w $\lambda$w.wyzx)} اعمال می‌شود.
				
				\item درون \lr{($\lambda$y.w $\lambda$w.wyzx)}، ابتدا \lr{$\lambda$w.wyzx} تعریف‌شده و سپس \lr{$\lambda$y} آن را به عنوان خروجی \lr{w} معرفی می‌کند.
				
			\end{itemize}
			
			بازنویسی کامل با پرانتزهای صریح:
			
			\[
			\left( \left( \lambda x . (x \, z) \right) \, \left( \lambda y . \left( w \, \left( \lambda w . \left( \left( \left( w \, y \right) z \right) x \right) \right) \right) \right) \right)
			\]
			
			\item \lr{$\lambda$x.xy $\lambda$x.yx} \par
			
			گام به گام:
			
			\begin{itemize}
				
				\item \lr{$\lambda$x.xy} تابعی است که بر \lr{($\lambda$x.yx)} اعمال می‌شود.
				
				\item این تابع دوم \lr{($\lambda$x.yx)} نیز خودش یک تعریف \lr{$\lambda$} است.
				
			\end{itemize}
			
			بازنویسی کامل با پرانتزهای صریح:
			
			\[
			((\lambda x.(xy))(\lambda x.(yx)))
			\]
			
			
			
		\end{enumerate}
		
	\section{روند کاهش $\beta$}
	
		کاهش $\beta$ در حساب $\lambda$ به معنای جای‌گذاری آرگومان‌ها در بدنه توابع و ساده‌سازی عبارات تا حد امکان است. این فرایند به صورت بازگشتی انجام می‌شود.
		
		\begin{enumerate}
			
			\item \lr{($\lambda$z.z)($\lambda$y.yy)($\lambda$x.xa)}
			
			\begin{itemize}
				
				\item {\large گام اول}: تابع \lr{($\lambda$z.z)} آرگومان اول خود را برمی‌گرداند.
				
				\[
				(\lambda z.z)(\lambda y.yy) \rightarrow (\lambda y.yy)
				\]
				
				\item {\large گام دوم}: عبارت به شکل زیر ساده می‌شود
				
				\[
				(\lambda y.yy)(\lambda x.xa)
				\]
				
				\item {\large گام سوم}: جای‌گذاری \lr{($\lambda$x.xa)} به‌جای y
				
				\[
				(\lambda y.yy)(\lambda x.xa) \rightarrow (\lambda x.xa)(\lambda x.xa)
				\]
				
				\item {\large نتیجه نهایی}:
				
				\[
				(\lambda x.xa)(\lambda x.xa)
				\]
				

				
			\end{itemize}
			
			عبارت دیگر قابل کاهش نیست زیرا تابع دیگر اعمال نمی‌شود.
			
			
			\item \lr{($\lambda$x.xx)($\lambda$y.yx)z}
			
			\begin{itemize}
				
				\item {\large گام اول}: ابتدا \lr{($\lambda$x.xx)} با آرگومان \lr{($\lambda$y.yx)} فراخوانی می‌شود. \lr{($\lambda$y.yx)} جای‌گذاری می‌شود:
				
				\[
				(\lambda x.xx)(\lambda y.yx) \rightarrow (\lambda y.yx)(\lambda y.yx)
				\]
				
				\item {\large گام دوم}: اکنون \lr{($\lambda$y.yx)} روی خودش اعمال می‌شود:
				
				\[
				(\lambda y.yx)(\lambda y.yx)
				\]
				
				\item {\large گام سوم}: تابع \lr{($\lambda$y.yx)} با آرگومان \lr{($\lambda$y.yx)} فراخوانی می‌شود. \lr{($\lambda$y.yx)} جای‌گذاری می‌شود:
				
				\[
				(\lambda y.yx)(\lambda y.yx) \rightarrow ((\lambda y.yx)x)
				\]
				
				\item {\large نتیجه نهایی}:
				
				\[
				((\lambda y.yx)x)
				\]
				
			\end{itemize}
			
			\item \lr{((($\lambda$x.$\lambda$y.(xy))($\lambda$y.y))w)}
						
			\begin{itemize}
				
				\item {\large گام اول}: ابتدا \lr{($\lambda$x.$\lambda$y.(xy))} با آرگومان \lr{($\lambda$y.y)} فراخوانی می‌شود. \lr{($\lambda$y.y)} جای‌گذاری می‌شود
				
				\[
				(\lambda x.\lambda y.(xy))(\lambda y.y) \rightarrow \lambda y((\lambda y.y)y)
				\]
				
				\item {\large گام دوم}: حالا \lr{($\lambda$y.(($\lambda$y.y)y))} با آرگومان \lr{w} فراخوانی می‌شود. \lr{y} به \lr{w} جای‌گذاری می‌شود
				
				\[
				(\lambda y((\lambda y.y)y))w \rightarrow ((\lambda y.y)w)
				\]
				
				\item {\large گام سوم}: تابع \lr{($\lambda$y.y)} روی \lr{w} اعمال می‌شود
				
				\[
				(\lambda y.y)w \rightarrow w
				\]
				
				\item {\large نتیجه نهایی}:
				
				\[
				w
				\]
				
			\end{itemize}
			
		\end{enumerate}
		
	\section{اثبات عبارات با استفاده از کدگذاری حساب $\lambda$}
	
		\begin{itemize}
			
			\item[الف.] \lr{not (not true) = true} \par
			تعاریف داده‌شده:
			\begin{itemize}
				
				\item \lr{not = $\lambda$x.((x false)true)}
				 
				\item \lr{true = $\lambda$x.$\lambda$y.x}
				
				\item \lr{false = $\lambda$x.$\lambda$y.y}
				
			\end{itemize}
			
			\begin{itemize}
				
				\item[گام 1:] محاسبه \lr{not true}
				
					\begin{itemize}
						
						\item تابع \lr{not} روی \lr{true} اعمال می‌شود:
						
						\[
						not true=(\lambda x.((x false)true))true
						\]
						
						\item حالا \lr{x} را با \lr{true} جای‌گذاری می‌کنیم:
						
						\[
						(\lambda x.((x false)true))true \rightarrow ((true false)true)
						\]
						
						\item تعریف \lr{true=$\lambda$x.$\lambda$y.x} را جای‌گذاری می‌کنیم:
						
						\[
						((true false)true)=((\lambda x.\lambda y.x)false)true
						\]
						
						\item حالا \lr{x} را با \lr{false} جای‌گذاری می‌کنیم:
						
						\[
						((\lambda x.\lambda y.x)false)true \rightarrow (\lambda y.false)true
						\]
						
						\item تابع \lr{($\lambda$.false)} را روی \lr{true} اعمال می‌کنیم:
						
						\[
						(\lambda y.false)true \rightarrow false
						\]
						
						\item نتیجه:
						
						\[
						not true=false
						\]
						
					\end{itemize}
				
				
				\item[گام 2:] محاسبه \lr{not (not true)}
				
					\begin{itemize}
						
						\item حالا \lr{not} را روی \lr{not true} که \lr{false} است اعمال می‌کنیم:
						
						\[
						not (not true)=(\lambda x.((x false)true))false
						\]
						
						\item \lr{x} را با \lr{false} جای‌گذاری می‌کنیم:
						
						\[
						(\lambda x.((x false)true))false \rightarrow ((false false)true)
						\]
						
						\item تعریف \lr{false=$\lambda$x.$\lambda$y.y} را جای‌گذاری می‌کنیم:
						
						\[
						((false false)true)=((\lambda x.\lambda y.y)false)true
						\]
						
						\item حالا \lr{x} را با \lr{false} جای‌گذاری می‌کنیم:
						
						\[
						((\lambda x.\lambda y.y)false)true \rightarrow (\lambda y.y)true
						\]
						
						\item تابع \lr{($\lambda$y.y)} را روی \lr{true} اعمال می‌کنیم:
						
						\[
						(\lambda y.y)true \rightarrow true
						\]
						
						\item نتیجه نهایی:
						
						\[
						not (not true)=true
						\]
						
					\end{itemize}
				
			\end{itemize}
			
			\item[ب.] \lr{if false then x else y = y}
			
			تعاریف داده‌شده:
			
			\begin{itemize}
				
				\item \lr{if a then b else c = a b c}
				
				\item \lr{true = $\lambda$x.$\lambda$y.x}
				
				\item \lr{false = $\lambda$x.$\lambda$y.y}
				
			\end{itemize}
			
			\begin{itemize}
				
				\item[گام 1:] جای‌گذاری در \lr{if false then x else y}
					
					تعریف \lr{if a then b else c} را استفاده می‌کنیم که برابر با \lr{abc} است. حال \lr{a=false}، \lr{b=x} و \lr{c=y}:
					
					if false then x else y=false x y

				\item[گام 2:] محاسبه \lr{false x y}
				
				\begin{itemize}
					
					\item تعریف \lr{false=$\lambda$x.$\lambda$y.y} را جای‌گذاری می‌کنیم:
					
					\[
					false x y=(\lambda x.\lambda y.y)xy
					\]
					
					\item ابتدا \lr{x} را در \lr{$\lambda$x.$\lambda$y.y} جای‌گذاری می‌کنیم:
					
					\[
					(\lambda x.\lambda y.y)x \rightarrow \lambda y.y
					\]
					
					\item حالا \lr{$\lambda$y.y} را روی \lr{y} اعمال می‌کنیم:
					
					\[
					(\lambda y.y)y \rightarrow y
					\]
					
					\item نتیجه نهایی:
					
					\[
					if false then x else y=y
					\]
					
					
				\end{itemize}
				
			\end{itemize}
				
		\end{itemize}
		
\end{document}